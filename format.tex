\usepackage{amsmath,amsfonts,amssymb}
\usepackage{graphicx}
\usepackage[plainpages=false,pdfpagelabels,bookmarks=true,backref=false,pagebackref=false]{hyperref}
\usepackage[tracking=true]{microtype}
\usepackage[nice]{nicefrac}
\usepackage{booktabs}
\usepackage{float}
\usepackage{flafter}
\usepackage{makeidx}
\usepackage{cite}
\usepackage{color}
\usepackage{wrapfig}
\usepackage{tabularx}
\usepackage{multicol}
\usepackage{paralist}
\usepackage{graphics}

\hypersetup{ 
	pdfauthor={\thesisAuthor},
	pdftitle={\thesisTitle},
	pdfsubject={\thesisType},
	colorlinks=false,
	breaklinks=true,
	pdfborder=0 0 0,
	citebordercolor=0 0 0,
	filebordercolor=0 0 0,
	linkbordercolor=0 0 0,
	menubordercolor=0 0 0,
	urlbordercolor=0 0 0,
	pdfhighlight=/I,
	bookmarksopen=false,
	bookmarksnumbered=true
}

\usepackage{listings}
\lstloadlanguages{C}
\lstset{numbers=left, numberstyle=\tiny, numbersep=5pt, language=C, basicstyle=\ttfamily\scriptsize, tabsize=2, breaklines=true}

% Disable single lines at the start of a paragraph (Schusterjungen)
\clubpenalty = 10000

% Disable single lines at the end of a paragraph (Hurenkinder)
\widowpenalty = 10000 \displaywidowpenalty = 10000

\renewcommand{\topfraction}{1.0}    % Bilder duerfen auch direkt am
                                    % Seitenanfang sein
\renewcommand{\bottomfraction}{1.0} % Bilder duerfen auch direkt am
                                    % Seitenende sein







% CUSTOM SECTION


%\makeatletter
%\newcommand{\figsourcefont}{\footnotesize}
%\newcommand{\figsource}[1]{%
%  \addtocontents{lof}{%
%    {\leftskip\cftfigindent
%     \advance\leftskip\cftfignumwidth
%     \rightskip\@tocrmarg
%     \figsourcefont#1\protect\par}%
%  }%
% }
%\makeatother

\newcommand{\highlight}[1]{\texttt{#1}}


\linespread{1.1}
\setlength{\parskip}{10pt}

\setcounter{secnumdepth}{3} % numbering of subsubsections
\setcounter{tocdepth}{3}	% includes subsubsections in the table of contents


\usepackage{booktabs}

% in place code
\lstnewenvironment{code}[1][]%
{
   \noindent
   \minipage{\linewidth} 
   \vspace{0.5\baselineskip}
   \lstset{basicstyle=\ttfamily\footnotesize,#1}}
{\endminipage}

% floating code
\lstnewenvironment{snippet}[1][]
    {\lstset{float=htpb,#1}} 
    {}

\usepackage{caption}
\captionsetup{labelfont={sf,bf},font=footnotesize,format=plain}

\usepackage{enumitem}

% Taken from Lena Herrmann at 
% http://lenaherrmann.net/2010/05/20/javascript-syntax-highlighting-in-the-latex-listings-package
\definecolor{lightlightgray}{rgb}{.95,.95,.95}
\definecolor{lightgray}{rgb}{.9,.9,.9}
\definecolor{darkgray}{rgb}{.4,.4,.4}
\definecolor{purple}{rgb}{0.65, 0.12, 0.82}

\lstdefinelanguage{JavaScript}{
	keywords={typeof, new, true, false, catch, function, return, null, catch, switch, var, if, in, while, do, else, case, break},
	keywordstyle=\color{blue}\bfseries,
	ndkeywords={class, export, boolean, throw, implements, import, this},
	ndkeywordstyle=\color{darkgray}\bfseries,
	identifierstyle=\color{black},
	sensitive=false,
	comment=[l]{//},
	morecomment=[s]{/*}{*/},
	commentstyle=\color{purple}\ttfamily,
	stringstyle=\color{red}\ttfamily,
	morestring=[b]',
	morestring=[b]"
}

\lstdefinestyle{nolinenumbers}
{
	numbers=none,
	backgroundcolor=\color{white}
}

\lstset{
%	language=JavaScript,
	backgroundcolor=\color{lightlightgray},
	extendedchars=true,
	basicstyle=\footnotesize\ttfamily,
	showstringspaces=false,
	showspaces=false,
	numbers=left,
	numberstyle=\footnotesize,
	numbersep=9pt,
	tabsize=2,
	breaklines=true,
	showtabs=false,
	captionpos=b
}


% END CUSTOM SECTION



\makeindex
