%!TEX root = ../thesis.tex

\chapter{First Appendix}
\label{sec:first_appendix}

\section{GitHub repositories}

All code developed in this project is published on GitHub and can be found at the following links.

\begin{itemize}
	\item{Code and documentation related to the local communication gateway running on a Raspberry Pi 2:\\ \url{https://github.com/spiegelm/smart-heating-local}}
	
	\item{Code and documentation related to the server:\\ \url{https://github.com/spiegelm/smart-heating-server}}
	
	\item{Code and documentation related to the mobile application:\\ \url{https://github.com/Octoshape/smart-heating-app}}
\end{itemize}

\section{Setup instructions for the local communication gateway}

This appendix explains the setup routine of the local communication gateways implemented on a Raspberry Pi 2.
See also the repository documentation:\\
\url{https://github.com/spiegelm/smart-heating-local}

\subsection{Setup raspbian}\label{setup-raspbian}

Flash raspbian to a SD card and boot the Raspberry. Find the IP address
using \highlight{nmap -sP [ip-adress]/[bitmask]}, e.g.
\highlight{nmap -sP 192.168.0.0/24}.

Open a SSH client and connect to the determined IP. The default username
and password are \highlight{pi} and \highlight{raspberry}. Type
\highlight{sudo raspi-config} to expand the filesystem, change the
password and set the local time zone.

Enable IPv6: Add \highlight{ipv6} on a line by itself at the end of
/etc/modules.

\subsection{Install dependencies}\label{install-dependencies}

Install \highlight{at}. Needed to remain tunslip6 started because UDEV
rules kill the spawning process.

\begin{verbatim}
sudo apt-get install at
\end{verbatim}

\subsubsection{Install Python 3.4.1 and	aiocoap}\label{install-python-3.4.1-and-aiocoap}

Credits to Marc Hüppin for the initial version.

openssl and libssl-dev are required for SSL support in python and is also required by pip.

\begin{quote}
	\highlight{sudo apt-get install sqlite3 libsqlite3-dev openssl libssl-dev}
	
	install the sqlite3 packages
	
	\begin{verbatim}
	mkdir ~/src
	cd ~/src
	wget https://www.python.org/ftp/python/3.4.1/Python-3.4.1.tgz
	\end{verbatim}
	
	unpack cd into dir
	
	\begin{verbatim}
	./configure
	make
	sudo make install
	\end{verbatim}
	
	get the aiocoap library from github: https://github.com/chrysn/aiocoap
	
	get setuptools from: https://pypi.python.org/pypi/setuptools
	
	install aiocoap using
	
	\begin{verbatim}
	python3.4 setup.py install
	\end{verbatim}
\end{quote}

\subsection{Setup smart-heating-local}\label{setup-smart-heating-local}

\subsubsection{Setup the border router connection}\label{setup-the-border-router-connection}

Clone this repository into your home folder:

\highlight{git clone https://github.com/spiegelm/smart-heating-local.git}.

Create symbolic links

\begin{itemize}
	\item{udev rules: \highlight{sudo ln -s /home/pi/smart-heating-local/rules.d/90-local.rules /etc/udev/rules.d/}}
	\item{tunslip executable:\\
		\highlight{sudo ln -s /home/pi/smart-heating-local/bin/tunslip6 /bin/}}
\end{itemize}

Add this line to \highlight{/etc/rc.local} to make sure the udev rule is also executed on startup

\noindent
\begin{minipage}{\linewidth}
	\begin{lstlisting}[numbers=none]
udevadm trigger --verbose --action=add --subsystem-match=usb --attr-match=idVendor=0403 --attr-match=idProduct=6001
	\end{lstlisting}
\end{minipage}

Reboot: \highlight{sudo reboot}

Attach the sky tmote usb dongle to the raspberry. The tun0 interface should be shown by \highlight{ifconfig}.
In case of problems run \highlight{sudo \textasciitilde/smart-heating-local/bin/\allowbreak start\_tunslip.sh}
manually.
Determine the ipv6 address of the Web service:
\highlight{less /var/log/tunslip6}:

\noindent
\begin{minipage}{\linewidth}
	\begin{lstlisting}[numbers=none]
Server IPv6 addresses:
fdfd::212:7400:115e:a9e5
fe80::212:7400:115e:a9e5
	\end{lstlisting}
\end{minipage}

Retrieve the registered routes on the border router:

\highlight{wget http://[fdfd::212:7400:115e:a9e5]}:

\noindent
\begin{minipage}{\linewidth}
\begin{lstlisting}[numbers=none]
<html><head><title>ContikiRPL</title></head><body>
Neighbors<pre>fe80::221:2eff:ff00:22d3
</pre>Routes<pre>fdfd::221:2eff:ff00:22d3/128 (via fe80::221:2eff:ff00:22d3) 16711422s
</pre></body></html>
\end{lstlisting}
\end{minipage}

Test route to the thermostat by requesting the current temperature via
coap-client (libcoap):

\highlight{\textasciitilde/smart-heating-local/bin/coap-client -m get\\
coap://[fdfd::221:2eff:ff00:22d3]/sensors/temperature}

Congratulations, your raspberry is connected to a thermostat!

\subsubsection{Install the required python packages}\label{install-the-required-python-packages}

Install pip: \url{https://pip.pypa.io/en/latest/installing.html}

\noindent
Install the project requirements:

\noindent
\highlight{cd ~/smart-heating-local/}\\
\highlight{pip install -r requirements.txt}

\subsubsection{Configure cron tasks}

Setup crontab to run the log and upload scripts periodically:

%\noindent
\highlight{crontab -e}

Insert these lines at the end of the file:

\noindent
\begin{minipage}{\linewidth}
\begin{lstlisting}
*/15 * * * * /usr/local/bin/python3.4 /home/pi/smart-heating-local/thermostat_sync.py
*/5 * * * * /usr/local/bin/python3.4 /home/pi/smart-heating-local/server_sync.py
\end{lstlisting}
\end{minipage}

These commands ensure that the temperature is polled from the registered
thermostats each 15 minutes and checked for uploading to the server each
5 minutes. The scripts log interesting events to
\highlight{\textasciitilde/smart-heating-local/logs/smart-heating.log}.


\section{Setup instructions for the server infrastructure}

Setup a virtual or dedicated server based on Ubuntu Linux 14.04 LTS and enter the following the commands in a terminal:

\noindent
\begin{minipage}{\linewidth}
	\begin{lstlisting}
# Install dependencies
sudo apt-get update
sudo apt-get upgrade
sudo apt-get install git python-virtualenv python3-pip
# Install code into home directory
cd ~
git clone https://github.com/spiegelm/smart-heating-server
# Setup virtualenv
virtualenv -p python3 env
. ~/env/bin/activate
# Check for python version >=3.4.0
python --version
# Install requirements, setup database and start server
cd ~/smart-heating-server/
pip install -r requirements.txt
./manage.py migrate
# Run server on port 8000
./manage.py runserver 0.0.0.0:8000
# Test that server is accessible via browser
# Kill server: CTRL-C
# Start server in the background using nohup
./scripts/restart_server.sh
# Kill background server
./scripts/kill_server.sh
	\end{lstlisting}
\end{minipage}

In case of software updates run the following commands to fetch the newest code form the repository and restart the server:

\noindent
\begin{minipage}{\linewidth}
	\begin{lstlisting}
cd ~/smart-heating-server
git pull
./scripts/restart_server.sh
	\end{lstlisting}
\end{minipage}



\section{Setup instructions for the mobile application}

\todo[inline]{SAMUEL: add setup instructions for mobile application}









