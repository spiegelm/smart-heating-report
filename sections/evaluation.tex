%!TEX root = ../thesis.tex

\chapter{Evaluation}
\label{sec:evaluation}

In this section we will evaluate our set design goals and discuss our implementation in detail.
Further we will analyze our design decisions.
Where applicable the functional and non-functional requirements will be validated.

\section{Infrastructure}

This section follows the same structure as used in the previous Chapter~\ref{sec:infrastructure}.
The infrastructure consists of the server part and the local part and both are evaluated in the following sections.

\subsection{Server Infrastructure}

\subsubsection{Design Goals}

Section~\ref{sec:server_infrastructure_design_goals} lists the design goals defined for this project part.
Each design goal will be evaluated in a separate paragraph below.

\paragraph{Modularity and Extensibility} Django emphasizes reusability of components.
This solid background allows us to clearly separate different concerns to achieve modularity.
Further the distinct components simplify application extensions by easily adding new resources, representations, views or URLs.

% TODO add some examples?

\paragraph{Usability} In order for a developer to be able to familiarize himself with a new API it is important to provide clear documentation and a platform to use the API.
The browsable API as described in Section~\ref{sec:server_infrastructure_restful_api} provides both.
It allows a developer to interactively explore the resource structure by navigating the hyperlinked URLs.
The browsable API also provides easy interaction possibilities to create, read, update and delete resources.
Further the API describes the allowed methods when requesting a particular resource.
One of these methods named \emph{OPTIONS} additionally shows fields and their requirements.

\paragraph{Testability}



Siehe auch Report Structure.


\paragraph{Design Decisions}

Django is designed for flat URLs but is still flexible enough to support the design decision of nested resources represented within the URL.


\subsection{Local Deployment}

Raspi: 27.7 10:30 - 15:45 ausgesteckt
27.7 21:45 - 05:30 development


\section{Mobile App}
