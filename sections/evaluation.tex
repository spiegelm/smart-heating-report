%!TEX root = ../thesis.tex

\chapter{Evaluation}
\label{sec:evaluation}

In this section we will evaluate our set design goals and discuss our implementation in detail.
Further we will analyze our design decisions.
Where applicable the functional and non-functional requirements will be validated.

\section{Infrastructure}

This section follows the same structure as used in the previous Chapter~\ref{sec:infrastructure}.
The infrastructure consists of the server part and the local part.
Both are evaluated in the following sections.

\subsection{Server Infrastructure}

\subsubsection{Design Goals}

Section~\ref{sec:server_infrastructure_design_goals} lists the design goals defined for this project part.
Each design goal will be evaluated in a separate paragraph below.

\paragraph{Modularity and Extensibility} Django emphasizes reusability of components.
This solid background allows us to clearly separate different concerns to achieve modularity.
Further the distinct components simplify application extensions by easily adding new resources, representations, views or URLs.

% TODO add some examples?

\paragraph{Usability} In order for a developer to be able to familiarize himself with a new API it is important to provide clear documentation and a platform to use the API.
The browsable API as described in Section~\ref{sec:server_infrastructure_restful_api} provides both.
It allows a developer to interactively explore the resource structure by navigating the hyperlinked URLs.
The browsable API also provides easy interaction possibilities to create, read, update and delete resources.
Further the API describes the allowed methods when requesting a particular resource.
One of these methods named \emph{OPTIONS} additionally shows fields and their requirements.

\paragraph{Testability}

The modular program structure and the chosen Django framework support the creation and automatized execution of software tests.
These automatized tests are continuously executed on a hosted service and show that the tests run regularly and successfully.

\todo[inline]{Siehe auch Report Structure.}


\subsubsection{Design Decisions}

During the system implementation a few design decisions were chosen and will be evaluated in the following paragraphs.

\paragraph{Nested URL schema}

The model hierarchy is represented using a nested URL schema.
This schema suits the model structure in tree form and induces good readable URLs.
Django is designed for flat URLs but is still flexible enough to support the design decision of nested resources represented within the URL.

\paragraph{Resource identifier as the first field}

Within the resource representation the resource identifier is always the first field.
This allows to easily recognize the identifier within resource representations and especially within nested resource representation.

\paragraph{Resource referencing}

Resources are referenced by including their representation.
This design decision improves the usability of the API, as related resources are shown in their full representation.
In contrast collections are referenced via their URL.
This is necessary to limit the representation size and avoid infinite recursions.
For example in a parent-child relationship the child representation contains the representation of its parent.
The parent representation itself contains the collection representation of its children.
Therefore the collection representation cannot contain the full representations of its resources.

\paragraph{Identification of URL fields}

Fields representing a URL are identified by the name \highlight{url} or the suffix \highlight{\_url}.
This design decision facilitates the recognition of hyperlinks which also allows an automatized navigation through the API.



\subsection{Local Deployment}


\subsubsection{Design Goals}

This section will evaluate each of the design goals defined in Section~\ref{sec:local_infrastructure_design_goals}.

The local infrastructure sections is about the implementation part done on the residential communication gateway.
Therefore the following paragraphs focus on the evaluation of software running on the Raspberry computer.

\paragraph{Performance}
% for low computational effort and power consumption on embedded systems.

The system performance depends on two aspects.
First, the software part.
The programming language \emph{Python} was chosen at it is suitable for developing on embedded hardware with possibly long-running input/output operations.
Another important component is the system design of independent scripts triggered by the scheduler \emph{cron}.
Second, the embedded hardware part.
The chosen hardware needs to be powerful enough to run the software implementation.

For evaluation of the ratio between required software performance and provided hardware performance the CPU load is monitored.
We use the \emph{sar} command from the \emph{apt-get} package \emph{sysstat} to analyze the average CPU utilization.
Over a period of seven days of regular system runtime the average CPU load was 0.28 percent.
This indicates that the hardware is powerful enough to run the designed software implementation.


\paragraph{Robustness}
% to let the system behave reasonably in presence of failures, especially connection problems.

We defined robustness to be the ability of the system to behave reasonably in the presence of failures, especially connection problems.
The local communication gateway is designed as two independent tasks.
The first task is the retrieval and storage of the temperature data and the application of the defined schedule.
The second task is the synchronization of the local data storage with the remote web server.
This design splits the "Aufgabenbereich" of the local communication system into two simpler separated tasks with a clearly defined interface, the local data storage.
In the presence of a failure only one of the two tasks is affected whereas the other task is still able continue its purpose.

For example in the event of a temporary internet connection disruption the system will keep operating the last downloaded heating schedule and store the temperature history.
After the disruption the local data storage is synchronized and the system continues to operate on the most actual settings.

Another failure scenario would be the temporary disconnection of a wireless thermostat.
In such a case the non-affected thermostats are still functional and the heating schedule of all thermostats is kept synchronized.
As soon as the system reestablishes the connection to the affected thermostat the system continues to work as planned.
In the meantime neither the properly operating thermostats nor the synchronization of all thermostats' heating schedules is impaired.


\paragraph{Reliability}
% for a high probability that the system operates as expected

In order to check the reliability of the system in a real world scenario the system has been deployed in a residential home.
Previous work in this area compared the defined room temperatures with the actual room temperatures to conclude on the proper work of a heating system\cite{eigenmann2012opportunisticSensing}.
Due to the season and the according temperatures this evaluation is not possible for this project.
Instead we focus on the communication between the wireless thermostats, the local communication gateway and the remote server.

The main data sources for our analysis are the generated log files and databases on the local communication gateway and the server infrastructure.

On the local communication gateway there are data sources: the log files written by two scripts and the local storage used to cache the temperature readings and other meta information.

\todo[inline]{Analysis of the local data storage}

% 2969 Rows returned from: SELECT * FROM heating_temperature
% WHERE timestamp BETWEEN '2015-08-01' AND '2015-09-01' (took 12ms)

% SELECT strftime('%d', timestamp), COUNT(*), * FROM heating_temperature
% WHERE timestamp BETWEEN '2015-08-01' AND '2015-09-01'
% GROUP BY strftime('%d', timestamp)
For the analysis of the local data storage we chose the time span of August 2015.
The local communication gateway queries the thermostats every 15 minutes, i.e. 96 times a day.
The maximal number of temperature measurements and meta entries would therefore be 2976 for the whole month of August.
The local data storage contains 2969 of these recorded temperature measurements in this date range, resulting in a coverage of 99.76 percent.
See Table~\ref{table:evaluation_local_database_coverage} for the full coverage data.

\begin{table}
	\begin{center}
		\begin{tabular}{ l | r r r }
			Local Database entries & Maximal count & Actual count & Coverage \\
			\hline
			Temperature					& 2976 & 2969 & 99.76 \% \\
%			Server temperature entries	& 2976 & 2969 & 99.76 \%
			Meta information			& 2976 & 2970 & 99.80 \% \\
		\end{tabular}
		\caption{Retrieved temperature measurements and meta entries in a real world deployment in August 2015.}
		\label{table:evaluation_local_database_coverage}
	\end{center}
\end{table}

\todo[inline]{Analysis of the local log files}

Each log entry has a severity level indicating how important a log event is in terms of system functionality and reliability.
We analyze these severity levels and their according log messages to show the intended system behavior as well as to identify potential problems.



The evaluation of the remote web server is similar to the analysis of the local log files and database.

\todo[inline]{Analysis of the server log files}

\todo[inline]{Analysis of the server database}
% 2969 Rows returned from: SELECT * FROM smart_heating_temperature WHERE thermostat_id = "04B753B9212580"
% AND datetime BETWEEN '2015-08-01' AND '2015-09-01' (took 23ms)






%Raspi: 27.7 10:30 - 15:45 ausgesteckt
%27.7 21:45 - 05:30 development




\paragraph{Interoperability}
% to cooperate with other distributed systems

We define interoperability as the system's ability to cooperate with other distributed systems.
The whole infrastructure is designed to communicate according to recognized open standards and architectural styles such as \emph{Constrained Application Protocol (CoAP)}, \emph{Hypertext Transfer Protocol (HTTP)}, \emph{Internet Protocol (IP)}, \emph{JavaScript Object Notation (JSON)} and \emph{Representational State Transfer (REST)}.
This way the developed infrastructure is able to cooperate with other distributed system via the defined interfaces.










\section{Mobile App}
