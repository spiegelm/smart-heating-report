%!TEX root = ../thesis.tex

\chapter{Evaluation}
\label{sec:evaluation}

In this section we will evaluate our set design goals and discuss our implementation in detail.
Further we will analyze our design decisions.
Where applicable the functional and non-functional requirements will be validated.

\section{Infrastructure}

This section follows the same structure as used in the previous Chapter~\ref{sec:infrastructure}.
The infrastructure consists of the server part and the local part.
Both are evaluated in the following sections.

\subsection{Server Infrastructure}

\subsubsection{Design Goals}

Section~\ref{sec:server_infrastructure_design_goals} lists the design goals defined for this project part.
Each design goal will be evaluated in a separate paragraph below.

\paragraph{Modularity and Extensibility} Django emphasizes reusability of components.
This solid background allows us to clearly separate different concerns to achieve modularity.
Further the distinct components simplify application extensions by easily adding new resources, representations, views or URLs.

% TODO add some examples?

\paragraph{Usability} In order for a developer to be able to familiarize himself with a new API it is important to provide clear documentation and a platform to use the API.
The browsable API as described in Section~\ref{sec:server_infrastructure_restful_api} provides both.
It allows a developer to interactively explore the resource structure by navigating the hyperlinked URLs.
The browsable API also provides easy interaction possibilities to create, read, update and delete resources.
Further the API describes the allowed methods when requesting a particular resource.
One of these methods named \emph{OPTIONS} additionally shows fields and their requirements.

\paragraph{Testability}

The modular program structure and the chosen Django framework support the creation and automatized execution of software tests.
These automatized tests are continuously executed on a hosted service and show that the tests run regularly and successfully.

\todo[inline]{Siehe auch Report Structure.}


\subsubsection{Design Decisions}

During the system implementation a few design decisions were chosen and will be evaluated in the following paragraphs.

\paragraph{Nested URL schema}

The model hierarchy is represented using a nested URL schema.
This schema suits the model structure in tree form and induces good readable URLs.
Django is designed for flat URLs but is still flexible enough to support the design decision of nested resources represented within the URL.

\paragraph{Resource identifier as the first field}

Within the resource representation the resource identifier is always the first field.
This allows to easily recognize the identifier within resource representations and especially within nested resource representation.

\paragraph{Resource referencing}

Resources are referenced by including their representation.
This design decision improves the usability of the API, as related resources are shown in their full representation.
In contrast collections are referenced via their URL.
This is necessary to limit the representation size and avoid infinite recursions.
For example in a parent-child relationship the child representation contains the representation of its parent.
The parent representation itself contains the collection representation of its children.
Therefore the collection representation cannot contain the full representations of its resources.

\paragraph{Identification of URL fields}

Fields representing a URL are identified by the name \highlight{url} or the suffix \highlight{\_url}.
This design decision facilitates the recognition of hyperlinks which also allows an automatized navigation through the API.



\subsection{Local Deployment}


\subsubsection{Desing Goals}

This section will evaluate each of the design goals defined in Section~\ref{sec:local_infrastructure_design_goals}.


\paragraph{Performance}
% for low computational effort and power consumption on embedded systems.

\paragraph{Reliability}
% for a high probability that the system operates as expected

Raspi: 27.7 10:30 - 15:45 ausgesteckt
27.7 21:45 - 05:30 development

\paragraph{Robustness}
% to let the system behave reasonably in presence of failures, escpecially connection problems.

\paragraph{Interoperability}
% to cooperate with other distributed systems










\section{Mobile App}
