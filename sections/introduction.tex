%!TEX root = ../thesis.tex

\chapter{Introduction}
\label{sec:introduction}

% Motivate your problem and outline the contributions of the thesis.\cite{mattern2010ict}

%As energy consumption is growing world wide, the need for technologies to control and analyze energy consumption increases.
%Therefore a lot of tools exist, that show the energy consumption of a household, business or certain devices. 
%But most of these technologies and tools are very complex for the end user, as most of the times the user is not a domain expert.

The world wide energy consumption is growing and causes increasing costs and environmental damage.
Private household contribute considerable to this trend.
In 2013, private households in Switzerland consumed 29\% of the country's total energy\cite{schweizerischeGesamtenergiestatistik2013}, most of it being used for heating purposes\cite{analyseEnergieverbrauchVerwendungszwecke2013}.
A lot of different tools in the domain of residential heating control were developed to help reduce the consumed energy for space heating.
But most of these tools do either not motivate end users to save energy or are too complex for the end user and thus get discarded in the long-term view.\todo[inline]{Why? References? Ihr solltet hier auf jeden Fall nochmal mit einem Paragraph auf das Problem eingehen. So ist die Arbeit nicht genuegend motiviert. Schaut euch eventuell mal die Einleitung von meiner Diss an und holt euch da ein paar Referenzen.}
This results in the need of user friendly tools that provide the end user with just enough information to analyze their data but not overwhelm him with unnecessary details.

%One of the areas in which easy to use tools are needed, is heating control.
%For this specific area we developed our project, which consists of two parts: the infrastructure and the mobile application.
%The infrastructure part gets the heating data and analyzes it while the mobile application provides an easy to use and well understandable visualization of the data.

For this specific area we developed our project, which consists of an infrastructure serving as a base, and a distributed mobile application allowing the user to configure and interact with the heating system.

This report is structured as follows:
%Chapter~\ref{sec:relatedwork} outlines previous work done in fields related to this thesis.
Chapter~\ref{sec:requirements} treats the process of determining the requirements of the infrastructure and applications developed in the scope of this lab project.
Chapter~\ref{sec:systemoverview} gives an overview of the whole system environment and introduces the chosen architectures of our application.
Chapter~\ref{sec:infrastructure} explains the developed infrastructure with the local deployment and the communication server in detail.
Chapter~\ref{sec:mobile_app} describes the implemented mobile application in detail.
In Chapter~\ref{sec:evaluation} we evaluate the developed system considering the defined requirements and design goals.



