%!TEX root = ../thesis.tex

\chapter{Requirements Elicitation}
\label{sec:requirements}

\todo[inline]{SAMUEL}

\section{Functional Requirements}

Functional requirements are stated using use cases in the style of Martin Fowler:
\todo[inline]{Add 2 sentences describing this style and put the links in footnotes or references.}
\url{http://en.wikipedia.org/wiki/Use_case#Martin_Fowler}\\
\url{http://ontolog.cim3.net/cgi-bin/wiki.pl?UseCasesMartinFowlerSimpleTextExample}

\subsection{Register a Residence}
\todo[inline]{Add one to two sentences explaining what the user wants to achieve (i.e. user has just bought the system and wants to enable smart heating in her residence.}
\begin{enumerate}
    \itemsep0em
    \item User opens the app and opens the registration screen
    \item User scans RFID tag
    \item System checks if RFID is not yet registered
    \item System registers residence with scanned RFID
    \item System shows empty home screen with no rooms
\end{enumerate}

\paragraph{Alternative} RFID already registered
\begin{itemize}
    \itemsep0em
    \item At step 3, system fails to verify that RFID is not yet registered
    \item{If RFID is associated to a residence
        \begin{itemize}
        \item System shows the home screen of the registered residence
        \end{itemize}
        }
    \item{If RFID is associated to a thermostat
        \begin{itemize}
        \item System shows ???
        \end{itemize}
    }
\end{itemize}
