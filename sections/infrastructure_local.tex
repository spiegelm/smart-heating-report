%!TEX root = ../thesis.tex

\section{Local Deployment}
\label{sec:local_infrastructure}

The local deployments consists of the residential communication gateway and the deployed thermostats with their wireless adapters.
The communication gateway collects the data read from the thermostats and sends it to the remote web server.
The thermostats are programmable and allow us to modify their behavior by flashing custom firmware.
This project uses the work of previous lab projects as a basis to build upon.
The primary focus is to improve the basic functionality of the communication gateway and create an unified but loosely coupled infrastructure by using the RESTful API provided by the server.
See also Figure~\ref{fig:residence_layout} for an overview of the local deployment.

\begin{figure}[h]
	\begin{center}
		\includegraphics[width=0.7\textwidth]{images/residence_layout_schema.png}
	\end{center}
	\caption{Example of a residence layout depicting a possible deployment. The local communication gateway is installed in the hallway, connected to the internet and has wireless connections to the deployed thermostats represented as antennas. Source of the original image: \url{http://www.haus-topplicht.de/wp-content/uploads/2013/12/planwohnung2.jpg}}
	\label{fig:residence_layout}
\end{figure}

\subsection{Existing Infrastructure}

This project builds upon work previously done by Nico Eigenmann as his master thesis \cite{eigenmann2012opportunisticSensing}.
Part of his work consisted of extending the hardware and software of the digital thermostat HR-20\footnote{\url{http://www.homexpertbyhoneywell.com/en-DE/Products/rondostat/Pages/HR-20.aspx}}, as depicted in Figure~\ref{fig:honeywell_hr20}, to offer wireless access via 6LoWPAN\footnote{Acronym of IPv6 over Low power Wireless Personal Area Network}.
Further a border router translates the 6LoWPAN messages into IPv6 packets and vice versa.
This border router is connected per Universal Serial Bus (USB) port to a computer and communicates via Serial Line Internet Protocol (SLIP).
The computer redirects the received messages into the connected Local Area Network (LAN) and also routes packets addressed to a sensor in the 6LoWPAN network via the border router.

\begin{figure}[h]
%\begin{wrapfigure}{r}{0.45\textwidth}
	\begin{center}
		\includegraphics[width=0.4\textwidth]{images/hr20.jpg}
	\end{center}
	\caption{Honeywell Rondostat HR-20 programmable thermostat. Image source: \url{https://piontecsmumble.files.wordpress.com/2013/02/hr20.jpg}}
	\label{fig:honeywell_hr20}
\end{figure}

\subsection{Design Goals}

Several design goals are determined in order to evaluate the system design and implementation.

\begin{itemize}
	\item \emph{Performance} for low computational effort and power consumption on embedded systems.
	\item \emph{Reliability} for a high probability to operate as expected.
	\item \emph{Robustness} to let the system behave reasonably in presence of failures.
	\item \emph{Interoperability} to cooperate with other distributed systems.
\end{itemize}

\subsection{Software Platform and Frameworks}

This sub section lists and describes the software components used in the local deployment.

\paragraph{tunslip6} is a tool to route IPv6 packets between a host and a border router via a serial line.
It is used to create a tunnel interface an the host system which acts as a regular network interface allowing

\paragraph{Python} is a multi-paradigm programming language and is suitable for desktop as well as for embedded systems.
This allows us to use a single programming language for the implemented software parts in the whole local infrastructure.
Further Python includes packages to facilitate asynchronous input and output which is required to concurrently query multiple network resources.

\paragraph{Constrained Application Protocol (CoAP)} \cite{rfc7252} is a application protocol created for low power devices and sensors that are heavily restricted in terms of computing power, memory size and power consumption.
Unlike the Hypertext Transfer Protocol (HTTP) commonly used in desktop and mobile systems CoAP is based on minimalistic network protocols to reduce overhead and computational requirements.

\paragraph{aiocoap} is a third-party package for Python implementing the CoAP protocol.

\paragraph{requests} is a library written in Python providing a simple and well designed interface to send HTTP requests.
This is used to communicate with the API provided by the Smart Heating Server.

\subsection{Implementation}

The communication gateway collects, caches and processes the data read from the thermostats as also the control commands from the server. The local communication gateway works as a proxy server and enables the local deployment to operate independently from the connection to the remote server. This way the last downloaded heating schedule is kept and operated until the server connection is be reestablished.
%Die grundlegende Einheit jedes Deployments ist die Residence. Eine Residence entspricht genau einem installiertem lokalen System, das die gelesenen Daten der Thermostate sowie Steuerbefehle des Servers sammelt, cached and ausführt. Das lokale Gerät arbeitet als ein lokales Gateway und sorgt dafür, dass der lokale Teil unabhängig von der Verbindung mit dem remote Server funktioniert.
% Temperaturen und andere Meta-Daten von den angebundenen Thermostaten sammelt und cached.

\todo[inline]{Overview picture the used local deployment: thermostat, radio module powerd by USB, Raspbeery with border router}
Honeywell HR-20, Radio Module deRFmega128, USB powered

